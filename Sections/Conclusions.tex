
This study utilized temperature-logging borehole datasets and satellite TIR data to clarify spatial variability of geothermal energy in Hokkaido Island, Japan. Distribution of subsurface temperature was molded from the borehole database using universal kriging technology, while LST anomalies were detected from the Landsat-8 band 10 of Thermal Infrared Sensor. UK clearly characterized the crustal temperature pattern which presented that high temperature were distributed in most western region and part of eastern region while low temperature were found in central region. The zones with the high geothermal potential are exactly coincident with the distribution of Quaternary volcanic rocks. In addition, LST result shows several temperature anomalies despite the fact that urban heat island phenomenal and elevation have a strong impact on the surface temperature. Combining with the analysis of subsurface heat distribution according to borehole temperature modeling and geological, the high geothermal potential areas were detected. The results of our work suggest that the combination of borehole data and TIRS remote sensing could be a simple and efficient way for geothermal exploration. However, the influence of the urban heat island and elevation should be better studied in the future research, and finally a heat flow map is expected to be conducted based on the combination approach.

