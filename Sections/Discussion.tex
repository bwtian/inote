\subsection{geological interpretation of the crustal temperature distribution}

Borehole data provide valuable information related to the subsurface thermal, geological structure of the geothermal systems. Our results show evident contrast of subsurface temperature among the three regions in Hokkaido; highest temperature in the western region and lowest ones in the central region. The subsurface temperature distribution modeled in our study is coinciding with the terrestrial heat flow contours in Hokkaido calculated by Ehara \cite{Ehara1972}. Particularly, the highest crustal temperature region at 1500 m b.s.l in western region (Fig.) was exactly consistent with the extremely high heat flow region as in \cite{Ehara1972}. Moreover, according to the modeling results of temperature at different depths, higher temperature occurs at shallower depth (e.g. 200 °C appears at depth of 500 b.s.l. ) in western region and part of eastern region, which reveals that the heat sources in such regions might be shallower. Ehara \cite{Ehara1972} inferred that temperature in western region amounted to 1000 °C at depth of 15-20 km but only 200 °C at the same depth in central region based on a series of assumption. In terms of the volcanology, the high subsurface temperature in western region of Hokkaido is corresponding with the “green tuff” areas, which have been centers of violent igneous activity since the beginning of the Neogene period \cite{Minato1956}. On the other hand, the high temperature in eastern region is also attributed to igneous activity which resulted from ocean plate subduction. 
\begin{figure} [ht!]
  \centering
    \includegraphics[width=0.7\textwidth]{Figs/f09_sub}
  \caption{Box-and-whisker plot of all the boreholes depth. Rectangle colored by green indicates the box-and-whisker plot and black points represent each borehole logging data. Blue cross shows the mean temperature value of boreholes}
  \label{fig:3d}
\end{figure}
The Hokkaido Island is situated in a subduction zone in the northwestern margin of the Pacific Ocean where the Pacific Plate is diving into island arcs. Magma, resulting from the pacific plate subducting, was produced under island arcs and presents the volcanic front, which runs nearly east-west from the eastern region through the central region and turns to the south in the western region (Fig.). In view of these, the northern side of volcanic front in central region should have high temperature. However, the results showed extremely low temperature at the site. Considering geological structure, there are two narrow metamorphic zones around the central region; one is Hidaka metamorphic belt and the other one is Kamuikotan belt (\url{http://www.glgarcs.net/figurepage/fig_basement_map.html}), where the geothermal activities characteristic of the island arc are interrupted \cite{Ehara1972}. Moreover, thick sedimentary layer (approximately 3-9 km), where heat production by radioactive heat sources is low \cite{Kametani1964, Minato1956}, might contribute to low temperature in the central region.
	
\subsection{LST anomalies and geothermal potential }

Several studies utilized MODIS LST to validate the LST retrieved from Landsat series remotely images \cite{Qin2011,Srivastava2009}. Srivastava \cite{Srivastava2009} discovered that the difference of LST from MODIS and ETM+ was no more than 2 °C, which supported our comparison results. Accordingly, the accuracy of the retrieved LST in our study is acceptable for geothermal detection.

Land surface temperature is mainly generated from solar radiation and Earth’s interior heat. However, in local area, land surface temperature is primarily influenced by soil heat flux \cite{Guoweidong2002}. The underground heat breaks the balance of surface energy and finally causes geothermal anomalies by changing the soil heat flux when it is transferred to the land surface through conduction and convection. Consequently, LST anomalies where the temperatures are higher than the background, with existence of underground heat source and availability of thermal channels, indicate high potential of geothermal resource. From the previous discussion, we have clarified the locations where heat source might exist, which is helpful to discriminate the LST anomalies of geothermal area from urban heat island. The current study area is extensive; as a result, regional LST is undoubtedly affected by various land-use and land-cover. In particular, urban heat island effects strongly increase the surface temperature. Although they present high LST anomaly, such areas should be excluded from areas of geothermal potential. Taking the location of heat source in mind, we examined four zones with evident LST anomalies which were located in the high subsurface temperature areas. According to the geological map, these four zones are all faulted area, which serves as thermal channel for heat transfer. With these two available prerequisites; heat source and thermal channel, we concluded that the four zones are potential areas of geothermal. In fact, we checked the four zones with Google earth software and found that two hot springs named Usubetsu and Marukoma are situated in zone B and zone C respectively, as shown in figure 9. On the other hand, zone D is located in OAkan-dake volcanic area. As for zone A, we did not find any manifestations except that a M4.2 earthquake occurred on April 16, 2004 19:39:22 UTC within this zone. Accordingly, we validated that these four zones are abound to be geothermal areas. Therefore, we believe that LST anomalies retrieved from Landsat-8 TIRS combined with subsurface temperature modeling are practical and effective way to examine geothermal potentials.  