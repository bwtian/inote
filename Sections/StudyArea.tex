
In this study, Hokkaido Island, the northernmost main island in Japan, was chosen as the study area because of relatively much amount of the temperature logging borehole data (Fig.~\ref{fig:hkd}). Hokkaido was formed after the Middle Miocene by the collision of the Kuril Arc and the Northeast Japan Arc at the central part of Hokkaido. Hokkaido therefore consists of three parts: the eastern region (part of the Kuril Arc), the central region (an arc-arc collision zone), and the western region (part of the Northeast Japan Arc). The volcanic front runs nearly east-west from the eastern region through the central region and turns to the south in the western region. This front line is parallel to the Kuril Trench and the Japan Trench. Hokkaido has broad gentle landforms. The mountains in Hokkaido generally have gentle slopes in the northern part and steep slopes in the southern part. Like many areas of Japan, Hokkaido is seismically active. Aside from numerous earthquakes, the following volcanoes are still considered active (at least one eruption since 1850): Mount Koma, Mount Usu and Showa-shinzan, Mount Tarumae, Mount Tokachi, Mount Meakan.
