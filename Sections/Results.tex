
\subsection{Subsurface temperature modeling}

The detailed data of subsurface temperatures from the surface to deep regions were obtained by temperature logging data. To characterize the spatial pattern of temperature, we chose universal kriging as an estimator to interpolate the borehole data. In the first place, a series of cross-validations were conducted to estimate the interpolation accuracy, which showed a high correlation coefficient of 0.92, 0.97, 0.93, 0.95, 0.98, 0.95, 0.95, 0.98, respectively between the original temperature data and predicted kriging value at sixteen depths, from 0 to 1500 m below sea level (m b.s.l.) with the interval of 200 m. UK satisfactorily characterized the spatial pattern of the temperature as shown in Fig. 6, in which distribution maps of the temperatures at sixteen depths were presented. 

The temperature distribution in the crust shows contrast spatially at all depths over the study area. However, the contrast is not obvious at shallow depth, such as 100 m b.s.l.. The regional contrast is more evident when the depth is deeper. The remarkable characteristic is that temperature in the western region (Northeast Japan Arc part) has widespread high values while temperature in the central region (an arc-arc collision zone) is low entirely. Central of the eastern region (Kuril Arc part) present relative higher value than the surrounding part, and such characteristic is more evident at greater depths, e.g. 1400 m b.s.l.. Comparatively, temperature in western region is higher than that in other two regions. In particular, extreme high crustal temperatures were presented with ellipse in Fig. Considering the geological setting, high temperature underground in Hokkaido is fairly coincident with the distribution of Quaternary volcanic rocks areas.  
\begin{figure} [ht!]
  %\centering
    \includegraphics[width=1\textwidth]{Figs/f07_3d}
  \caption{Box-and-whisker plot of all the boreholes depth. Rectangle colored by green indicates the box-and-whisker plot and black points represent each borehole logging data. Blue cross shows the mean temperature value of boreholes}
  \label{fig:3d}
\end{figure}
 
\begin{figure} [ht!]
  %\centering
    \includegraphics[width=1\textwidth]{Figs/f08_lst}
  \caption{Box-and-whisker plot of all the boreholes depth. Rectangle colored by green indicates the box-and-whisker plot and black points represent each borehole logging data. Blue cross shows the mean temperature value of boreholes}
  \label{fig:lst}
\end{figure}

\subsection{LST}

Before determining the surface temperature characteristics retrieved from landsat-8 TIRS, we need to validate the capabilities of the method. To this end, MODIS LST data were utilized to be compared with LST retrieved from the TIRS because of the unavailability of in situ data. The comparison shows that the average temperature differences are all within 2 k. The result of combined LST anomaly was shown in Fig. 8. Regionally, the extreme high LST with red color are almost coinciding with areas affected by human activity including area of urban, patty and crop, of which urban areas have the highest LST. On the other hand, the lowest LST present in areas with high elevation where forests are dominant. Aside from all the high LST areas that influenced by human activity, four cases, with A, B, C, D denoted in Fig.8, were chosen to examine their possibility of being of geothermal potential. These four zones are all situated at areas with high underground temperature. More details of these four zones were shown in Fig. 9. Additionally, the corresponding LULC in these four zones are almost uniform (Fig.9), excluding the LULC effects on the LST. The results show that all these four zones are characterized with evident LST anomalies (Fig.9). According to the results, the high temperatures in the four zones are all 6-12 °C higher than the background. 