% %2.2 data resources and processing
% % % Figure 2 %trim option's parameter order: left bottom right top



\subsection{Volcanoes data}

Quaternary Volcanoes are Volcanoes which were active during the Quaternary Period (ca. 2.6 Ma to present), including active volcanoes, which are the ones that have erupted within the past 10,000 years or volcanoes with vigorous fumarolic activity. The Quaternary Volcanoes data, which were provided by the website of Geological Survey of Japan (GSJ), The National Institute of Advanced Industrial Science and Technology (AIST) (\url{https://maps.google.co.jp/maps?q=https://gbank.gsj.jp/volcano/Quat_Vol/kml/volcano20131204a_en.kml}), was acquired with the keyhole markup language (KML) format. The active volcanoes data was acquired from website of Japan Meteorological Agency (\url{http://www.seisvol.kishou.go.jp/tokyo/STOCK/souran/appendix/active\_volcanoes.zip}), also in the format of KML. There are 78 Quaternary Volcanoes (orange cross triangle in Figure~\ref{fig:hkd}) located on main island of Hokkaido, of which 18 ones are categorized as active volcanoes(red triangle in Figure~\ref{fig:hkd}).
% % % Figure 3
Figure 3 3D presentations of all the boreholes with temperature logs.

\begin{figure} [ht!]
  %\centering
    \includegraphics[width=1\textwidth]{Figs/f03_bhTD.pdf}
  \caption{Presentations of Temperature profiles in Hokkaido area with Depth. Red line showed the linear regression of all the boreholes temperatures, blue dot line shows the water's bolling temperature $100\,^{\circ}\mathrm{C}$}
  \label{fig:bh3d}
\end{figure}
Figure~\ref{fig:bh3d} shows the 3D presentations.

% % % Figure4 
\begin{figure} [ht!]
  %\centering
    \includegraphics[width=1\textwidth]{Figs/f04_box2}
  \caption{Box-and-whisker plot of temperature distribution in different boreholes depth. The blue diamond represents mean value of group and red point are outliers of temperature.
  }
  \label{fig:box2}
\end{figure}
Figure~\ref{fig:box2} shows the 3D presentations.

\subsection{Borehole Temperature}

The borehole dataset was acquired from the "temperatures profiles in Japan" datasets compiled by Japan Atomic Energy Agency in 2004 (\url{http://www.jaea.go.jp/04/tono/siryou/ welltempdb.html}). The borehole temperature logging data implemented in this study is a subset from the above datasets. After checking, processing and removing outliers of the borehole data \cite{Tian2014}, 28,476 temperature records in the 433 boreholes loggings in Hokkaido, with area of 83,453.57 km$^{2}$, were collected. Range depths of the boreholes over the study region vary from 241 m to 2200 m, of which 70 boreholes are less than 500 m, 129 boreholes are during 500 to 1000m, 182 boreholes are during 1000 to 1500 m, 51 boreholes are during 1500 to 2000m and 1 borehole is over 2000 m (Fig.~\ref{fig:hkd}). The 1st quartile, median, mean, 3rd quartile of total depth are 953m, 1181m, 1135m, 1500m respectively (Fig. ~\ref{fig:boxplot}). The three dimension distribution of subsurface temperature is represented in Figure 3. It was obviously that the temperature increase from surface with increasing the depth (Figs. 3 and 4). 
Figure 2 




% % % Figure LULC 
\begin{figure} [ht!]
  %\centering
    \includegraphics[width=1\textwidth]{Figs/f06_lulc}
  \caption{LULC data  }
  \label{fig:lulc}
\end{figure}
Figure~\ref{fig:lulc} shows the 3D presentations.

 

\subsection{Remote Sensing data}

The Landsat 8 satellite carries a two-sensor payload, the Operational Land Imager (OLI) and the Thermal Infrared Sensor (TIRS). The reflectance of Landsat-8 OLI and TIRS were measured in eleven spectral bands: coastal/aerosol (0.44-0.45μm) visible blue (0.45-0.51μm), green (0.53-0.59 μm), red (0.64-0.67 μm), NIR (0.85-0.88μm), SWIR (1.57-1.65 and 2.11-2.29 μm), cirrus (1.36-1.38 μm), Thermal infrared (10.6–11.19 μm and 11.5-12.51 μm) and panchromatic mode (0.5-0.68 μm). In this study, nine scenes of thermal infrared band 10 of Landsat-8 image with cloud cover less than 10\% was selected as the suitable thermal infrared data to derive in the Hokkaido areas from different seasons (Table 1).
In addition, the 50-m resolution Land-Use and Land-Cover (LULC) data which is consistent geographically with Hokkaido was retrieved from the JAXA EORC web site (\url{http://www.eorc.jaxa.jp/ALOS/lulc/lulc_jindex.htm}). The LULC data was generated using AVNIR-2, which were categorized as water, urban, paddy, crop field, grass land, deciduous tree, evergreen tree, bare land, snow and ice (Fig.\ref{fig:lulc}).
