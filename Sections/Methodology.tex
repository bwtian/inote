\subsection{workflow}
\subsection{LST}  

Many algorithms for the LST estimation from satellite TIR measurements have been proposed utilizing different assumptions \cite{Li2013a, Li2012}. In this study, we retrieved the LST from the Landsat 8 TIRS band 10 using the Semi-Automatic Classification Plugin in QGIS. Firstly, bands of At-sensor Temperature were converted from DN bands. Then, classifications of land cover were obtained to estimate the land surface emissivity. Finally, LST was calculated from at-sensor temperature and land surface emissivity using the method proposed by A et al. \cite{Artis1982}.

Because the landsat images were acquired in different seasons which have strong impact to surface temperature, a normalized method were conducted to relieve such effect. The LST was normalized by minus the mean temperature values of each image as follows: 
LSTn = LST – LSTm                             (1)
where LSTn and LSTm denote the normalized LST and mean LST of each Landsat-8 band10 image, respectively.  

\subsection{Validation Method}
3D modeling of subsurface temperature

Universal Kriging (UK) (Eq. 1) was chosen as an estimator to interpolate the well-loggings temperatures at different depths. UK is often used for the regionalized data with a significant spatial trend. UK is an extension of ordinary kriging (OK) by incorporating the local trend within the neighborhood search widow as a smoothly varying function of the coordinates, and then estimates the trend components within each search neighborhood window and performs simple kriging (SK) on the corresponding residuals.
where  is deterministic function with known function  and unknown function , L  N. λi is kriging weight; N is the number of sampled points used to make the estimation and depends on the size of the search window; and μ(x0) is the mean of samples within the search window.

Additionally, we are developing an add on R package for TIR data analysis and mapping(\url{https://github.com/bwtian/TIR}) \cite{R-TIR} using R programing language \cite{R}, and a suite of its add on packages (raster \cite{R-raster}, sp \cite{R-sp}, rgeos \cite{R-rgeos}, rgdal \cite{R-rgdal}, gdalUtils \cite{R-gdalUtils}, landsat \cite{R-landsat}, gstat \cite{R-gstat}, maptools \cite{R-maptools}, ggplot2 \cite{R-ggplot2}, ggmap \cite{R-ggmap}, plotKML \cite{R-plotKML}, rasterVis \cite{R-rasterVis}, and lattice \cite{R-lattice}) were utilized for data download, analysis and mapping.