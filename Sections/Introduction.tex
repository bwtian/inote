%\bibliography{p1402-cited}

Worldwide geothermal booms, and suspension of all nuclear reactors in Japan after the Great East Japan Magnitude 9.0 Earthquake, tsunami, and nuclear meltdown triple disasters on 11th March 2011, have encouraged researchers and policy makers to consider renewable energies more seriously. Renewable nature energy, such as geothermal, solar, wind and bio energy are widely accepted as key sources for the future, not only for Japan but also for the world. Geothermal energy is a kind of clean, sustainable, and renewable thermal energy resource generated and stored in the shallow crust, which has been exploited for decades to generate electricity (the high temperature uses), or space heating and industrial processes (the low-temperature uses) \cite{Barbier2002}. As a country located on the “pacific ring of fire”, Japan is blessed with much geothermal resources. There are nearly 200 volcanoes in Japan, of which 110 are active. It has the third ranking geothermal energy potential, and its geothermal electricity production is currently eighth in the world
[4]. Common to geothermal resource evaluation, exploration, development, and management, the accurate imaging of geothermal structure and spatial distribution from the ground surface down to great depths is beneficial but yet an interdisciplinary problem.
Geothermal energy is in huge quantities, but it is unevenly distributed and often at depths too great to be exploited. Data obtained from wells (i.e. drill cuttings, cores, and logs) are the most reliable and direct one for characterizing the thermal, geochemical and geological structures of geothermal systems \cite{Asaue2006, Teng2007}. In situ drilling boreholes data provide the most accurate information in the exploration process. Drilling is, however, the mostly costly exploration method and hence the number and depth of the wells tend to be limited and they also tend to be irregularly distributed throughout the study area. On the other hand, active geothermal fields have natural manifestations and temperature anomalies at the ground surface, such as volcanoes, hot springs, mud pots and fumaroles, which are natural indicators of geothermal activity that can provide a visible sign of the transport of heat and mass through the crust. At a preliminary stage of geothermal exploration, multiple geological, geophysical, and geochemical surveys are carried out in and around these surface manifestations. On the basis of this consideration, Thermal infrared remote sensing (TIR) which is an efficient technique for obtaining the land surface temperature (LST) has great potential to contribute positively to cost-effective geothermal energy evaluation and exploration. Consequently, combination of 3D modeling of subsurface temperature based on boreholes data and surface energy distribution retrieved from TIR could provide more valuable information for exploration of geothermal resources.

3D modeling of subsurface temperature using limited borehole data is useful for visualizing subsurface temperature distribution and hence is helpful in geothermal resources exploitation, since the temperature degree and depth of geothermal resources decide the method of their utilization (geothermal power plant or direct use). Kriging has been proved to be a suitable interpolation method \cite{Agemar2012} . Of all the Krigng types, we choose universal kriging, which makes it possible to accommodate a trend in data which is essential for the estimation of subsurface temperatures. 

LST anomaly is a key indicator of geothermal areas. The application of TIR for detecting and quantifying LST anomalies have been successfully used in geothermal fields since 1960s \cite{Qin2011}. Hellman \cite{Hellman2004} detected the hot springs using thermal infrared data of Thermal Emission and Reflection Radiometer (ASTER) and Airborne Visible/Infrared Imaging Spectrometer (AVIRIS). Benali et al \cite{Benali2012} using MODIS LST data to estimate air surface temperature. Sobrino \cite{Sobrino2004} compared three method of retrieving LST from band 6 of Thematic Mapper (TM) sensor onboard the Landsat 5 satellite. In particular, Landsat (TM/ETM+) are one of the most used for environmental studies \cite{Andres1995, Flynn2001, Harris1999, Wang2013}  Thermal Remote Sensing (TIR) was started from band 6 of TM onboard Landsat 4. The recent launch of the Landsat-8 satellite in February 2013 ensures the continuity of remote sensing data onboard previous Landsat series of satellites \cite{Jimenez-munoz2014}. In this current study, the Landsat 8 TIRS band 10 were used to retrieve the LST.

Understanding of surface energy balance and underground heat transfer will contribute to the identification of geothermal areas. Accordingly, the objective of this study is to present the spatial variability of geothermal energy distribution in Hokkaido, Japan, by comparison and combination of 3D modeling of well-logging temperature data with Land surface temperature (LST) data derived from TIR.

\begin{figure} [ht!]
  %\centering
    \includegraphics[width=1\textwidth]{Figs/f01_StudyArea}
  \caption{Presentation of study area, Hokkaido Island in northern Japan. Geographical distribution of borehole data (black dot) with depth, and active volcanoes (red triangle) were shown. Bottom depths are shown by the circles with different color and size.}
  \label{fig:hkd}
\end{figure}
Figure~\ref{fig:hkd} shows the location of study area.


\begin{figure} [ht!]
  %\centering
    \includegraphics[width=1\textwidth]{Figs/f02_geology}
  \caption{Geology settings of Hokkaido Area}
  \label{fig:geology}
\end{figure}


%In this paper we fouced on the method use newest Landsat serreis remote sensing
%Satellite Landsat 8.
%
%In the context of global warming, reduction of fossil energy related carbon
%dioxide ($CO_2$) emissions and worldwide anti-nuclear movements after triple disaster of Great East Japan magnitude 9.0 earthquake, and its triggered tsunami
%and nuclear meltdown on 11th March 2011 in Japan, energy with more characteristics of clean, green, renewable, safe, stable and sustainable is requisite. Geothermal energy is such kind of energy which from radioactively generated heat internal earth. The heat flow transfers from earth's inner core through crust and finally reach to the land surface through convection and conduction. Therefore, land Surface temperature (LST) anomalies where the temperature higher than the background can be used to detect areas with high potential geothermal energy. Compared with transitional geological, geophysical and geochemical surveys,  the application of thermal remote sensing (TIR) techniques in the initial phases of geothermal exploration plays a very cost-effective role.

